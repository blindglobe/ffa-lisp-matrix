\documentclass[12pt,a4paper,dvipdfm]{article}

\usepackage{amsmath}
\usepackage{amsfonts}
%\usepackage[letterpaper,vmargin=1.7in]{geometry}
%\usepackage[letterpaper,left=2cm,right=8cm,bottom=3cm,top=3cm,marginparwidth=4cm]{geometry}
%\usepackage{natbib}
\usepackage{graphicx}
\usepackage{url}
\usepackage{natbib}
\usepackage{color}
\usepackage{paralist}           % compactitem
\usepackage{hyperref}
\DeclareMathOperator{\E}{E}
\DeclareMathOperator{\Var}{Var}
\DeclareMathOperator{\Cov}{Cov}
\DeclareMathOperator{\argmin}{argmin}
\DeclareMathOperator{\argmax}{argmax}
\DeclareMathSymbol{\ueps}{\mathord}{letters}{"0F} % ugly epsilon
\renewcommand{\epsilon}{\varepsilon}
\newcommand{\aseq}{\overset{as}=} % almost surely equals
\definecolor{red}{rgb}{1,0,0}
\newcommand{\FIXME}[1]{{\color{red}\{FIXME: #1\}}}

\renewcommand\floatpagefraction{.9}
\renewcommand\topfraction{.9}
\renewcommand\bottomfraction{.9}
\renewcommand\textfraction{.1}

\usepackage{listings}
\lstset{
  language=Lisp,
  extendedchars=true,
  basicstyle=\normalsize\ttfamily,
  stringstyle=\ttfamily,
  commentstyle=\slshape,
%  numbers=left,
%  stepnumber=5,
%  numbersep=6pt,
%  numberstyle=\footnotesize,
  breaklines=true,
%  frame=single,
  columns=fullflexible,
}

\begin{document}

\title{The FFA library}
\author{Tam\'as K Papp (\url{tpapp@princeton.edu})}
\maketitle

\section{Introduction}
\label{sec:motivation}

Even though Common Lisp has extensive libraries, sometimes the need
arises to call functions written in other languages, especially C.
While CFFI provides a comfortable and unified interface for most
purposes, using functions that expect to find or output arrays at
locations specified by a pointer still doesn't have a common
interface.  Constructing an array at a memory location is always
possible by allocating a chunk of memory and copying the array
elements in and out manually, but some implementations provide direct
access to an unboxed array at a memory location (eg SBCL).

The FFA (Foreign Friendly Array) library provides an interface that
allows the user to map Lisp arrays into a specified memory location
for the body of a macro call.  This macro has well-defined semantics,
explained in Section~\ref{sec:with-pointer-array}, which is
implemented differently for various implementations to take advantage
of implementation-specific optimizations offered.

Note that the approach of this package is to keep arrays in Lisp, and
provide access to arrays mapped to a given pointer on demand,
incurring the possible overhead at the time.  A different approach is
taken by
\href{http://middleangle.com/rif/derifatives/Home/21/cl-blapack-alpha-release}{CL-BLAPACK},
which keeps vectors in foreign memory, where they are untouched by the
GC.  Each design choice has advantages and disadvantages.  Keeping the
arrays in Lisp was chosen because
\begin{itemize}
\item access to array elements is fast, several implementations
  optimize \lstinline!aref! quite well
\item existing code doesn't have to be made aware of some special
  array type and can continue to use plain Lisp arrays
\item foreign memory is usually more scarce than memory available in
  Lisp\footnote{This is relevant even in implementations that require
    copying.  Imagine that you have 100 matrices, and you are calling
    a foreign function that needs a few of them at a time.}
\end{itemize}

There are some disadvantages: (1) some implementations require
copying, introducing an overhead, (2) sometimes you need to have an
array that stays in one place for a long time (eg for callback
functions), and (3) Common Lisp arrays are row-major, while some
languages and libraries expect column-major arrays by default.

Regarding (1), FFA doesn't claim to be superfast in all
implementations --- it simply \emph{works}.  If you find that copying
is an unbearable overhead, you can either switch implementations, or
ask the authors of your implementation for arrays that can be pinned.
There is simply no way to make array access fast in every situation
without the right facilities.  For (2), the answer is that FFA is not
the right library for you: simply allocate a chunk of memory and free
it manually when you are done.

Regarding the row-major vs.~column-major issue: some foreign libraries
(eg BLAS, Atlas) accept both kinds of arrays, so there is no need to
transpose.  In the worst case, you can easily write a transpose
function in either C or Lisp, the latter is provided in this package.

\section{Flat arrays}
\label{sec:flat-arrays}

For the purposes of this package, flat arrays are arrays that have a
rank of one, ie are one-dimensional.  Flat arrays are particularly
important because most implementations that allow direct access to
unboxed arrays require that these arrays are simple-arrays of rank
one.  Fortunately, Common Lisp has displaced arrays, so you can always
create a flat array and make it the target of a displaced array of the
desired dimension.

This is what \lstinline!make-ffa! does.  It uses the following syntax:
\begin{lstlisting}
make-ffa dimensions element-type &key
         (initial-element 0 initial-element-p)
         (initial-contents nil initial-contents-p)
\end{lstlisting}
where all the arguments are the same ones that would be provided for
\lstinline!make-array!, except that \lstinline!initial-element! and
\lstinline!initial-contents! will be coerced to the desired type, and
\lstinline!element-type! (which is mandatory, not optional) is also
allowed to be one of
\begin{lstlisting}
(:int8 :uint8 :int16 :uint16 :int32 :uint32 :float :double)
\end{lstlisting}
in which case the corresponding Lisp type is chosen.  If
\lstinline!dimension! is a list with more than one element, a flat
array is created, and a displaced array is returned.
\lstinline!make-ffa! does not guarantee that the resulting array is
unboxed, because this is always implementation-dependent, but it
guarantees to try its best.

Even if you don't use foreign functions, arrays displaced to flat
arrays are quite handy.  For example, if we want to sum elements in an
array, we just find the original array (see
\lstinline!find-original-array!) and call reduce.  If the original
array is not flat, \lstinline!find-or-displace-to-flat-array! will
provide a displaced flat one, but benchmarks indicate that calling
reduce on this is not as efficient as calling reduce on a flat,
non-displaced array of the same dimension.

Some handy operations are written in CL and provided in
\verb!operations.lisp!, including \lstinline!array-reduce! (which has
an option for ``ignoring'' \lstinline!nil! values), and the derived
functions \lstinline!array-max!, \lstinline!array-min!,
\lstinline!array-sum!, \lstinline!array-product!,
\lstinline!array-count! and \lstinline!array-range! --- see the
functions for details.  You can get a generalized outer product using
\lstinline!outer-product!.  See the source code for more useful
operations.

You can make a copy of an array using \lstinline!array-copy!, which,
however, does not descend into the elements (eg if you are using this
array to store lists, the elements of the new array will reference the
same lists as the elements in the old one).  You can also map arrays
(elementwise) into arrays of the same dimensions (but not necessarily
the same element-type) with \lstinline!array-map!, which you can also
use for a deep copy with the appropriate function.  The function
\lstinline!array-convert! is a specialized version of
\lstinline!array-map!, converting array elements using
\lstinline!coerce!.


\section{with-pointer-to-array}
\label{sec:with-pointer-array}

The key macro is 
\begin{lstlisting}
(with-pointer-to-array (array pointer cffi-type length direction)
  &body body)
\end{lstlisting}

Its semantics are defined as follows.  Within the body of the macro,
the \lstinline!pointer! will be a CFFI pointer pointing to a
contiguous region in memory, which contains \lstinline!length!
elements of type \lstinline!cffi-type!.  If direction is either
\lstinline!:copy-in! or \lstinline!:copy-in-out!, the area is
guaranteed to contain the elements of the array at the beginning of
the body of the macro.  If direction is one of \lstinline!:copy-out!
or \lstinline!:copy-in-out!, the array is guaranteed to contain the
elements of the memory location after the body of the macro ends.

Note that \lstinline!cffi-type! doesn't have to match the element-type
of array.  It is advised that it does, because otherwise the macro
will try to coerce the elements to the desired foreign type before
copying, which can result in possible errors or an efficiency loss.

Also note that \lstinline!array! does not have to be a simple, flat or
unboxed array.  If the implementation can't provide direct access to
that array type, the elements will be copied.  The note above about
efficiency also applies here.

To make this more clear, in the example below an array is created
using \lstinline!make-array! with element-type \lstinline!integer!.
Then at runtime, elements are coerced to \lstinline!:int32! (if they
are small enough, which holds here).  This is not efficient, so in
SBCL, a warning is issued.

\begin{lstlisting}
(defun cffi-fill-int32 (pointer size)
  "Fill array at pointer with size integers."
  (dotimes (i size)
    (setf (cffi:mem-aref pointer :int32 i) i)))

(let* ((a (make-array '(2 9) :element-type 'integer))
	      (a-d (displace-array a 9 3)))
  (with-pointer-to-array (a-d pp :int32 7 :copy-in-out)
    (cffi-fill-int32 pp 7))
  a) ; =>  #2A((0 0 0 0 1 2 3 4 5) (6 0 0 0 0 0 0 0 0)))
\end{lstlisting}

If frequently happens that you have to pass more than one array to a
foreign function.  For those cases, you can use the convenience macro

\begin{lstlisting}
(with-pointers-to-arrays ((a a-pointer a-type a-length a-dir)
                             (b b-pointer b-type b-length b-dir)
                             ...)
   ...body...)
\end{lstlisting}
which expands into nested calls of \lstinline!with-pointer-to-array!
using the given parameters.

\end{document}

%%% Local Variables: 
%%% mode: latex
%%% TeX-master: t
%%% End: 
